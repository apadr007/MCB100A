
\documentclass{article}
\usepackage[utf8]{inputenc}

\title{MCB 100A}
\author{Alex Padron}
\date{26 February 2016}

\begin{document}

\maketitle

\section{Chapter 6}

Whenever you see an energy curve – think of hills and valleys in gravitational fields. It's useful to look & thikn of these curves in that way. 

\mbox{ }\\
more molecular systems have characteristic potential energy as a function of time. 

\mbox{ }\\
Where is the attractive force coming from? It's from polarization effects. The charges of atoms are flexible and if these clouds can move they can become entrained with each other, and that is what causes the attractive force between atoms. 

U_{bond} = D(1-e^-a(r-r_o)^2

\section{Harmonic approximation}

It's useful to simplify even further an interaction potential. A harmonic potential is something that goes with r^2 and nothing else. Every function can be described using the taylor series. 

\section{Grove's experiment}
Grove's measured the spacing of his membrane molecules for thousands of points–he's measuring the spacing between the black domains in the figure. He then fit a gassian to these data. He used boltzmann to extract the potential. 

p(r) = (a/pi*e^-a(r-r_o)^2)^1/2
p(r) = 1/Q*e^-U(r-r_o)/K_B*T
U(r) = K_B*Ta(r-r_0)^2 + constants

by taking the second derivative you can get out a harmonic spring constant

\section{The main part for today: last part of 6}

VDW interactions are commonly used bc it's very useful. 

VDW contacts: where molecules are loosely touching each other. 

\textbf{Coulomb's law }
    Take two charged particles and measure the force between them. 
    
    \mbox{ }\\
    If you're talking about electric charges, you have to talk about electric fields. 
    
    F_i = q_iE_j 
    
    \mbox{ }\\
    you can express the electric field similar to the charge
    
    E_i = \[\frac{1}{4\epsilon\epsilon_0} \frac{q_j}{r^2}\]
    
    \textbf{Electric Potential}
        
    \begin{itemize}
        \item work + force
        \item $\int_{a}^{b} x^2 dx$
        \item $\int qE dr$ = work done
    \end{itemize}
    
    \begin{itemize}
        \item work + force 
        \begin{enumerate}
            \item $\int_{a}^{b} x^2 dx$
        \end{enumerate}
        \item W = $\int qE dr$ 
    \end{itemize}
            
    It is the negative derivative of the electric field
    
    \mbox{ }\\
    The very subtle and small attractive interactions for every charge with other molecules is through a polarization effect and that's the dielectric constant. The average inside a protein is 2. 
    
    \mbox{ }\\
    The fact that water can rotate gives it a much larger polarizability–almost two orders of magnitude reduction in the electric field. Water will also have mobile charges in it! Therefore its a conductor full of charges that can move. 
        
        


\end{document}
